\section{Experimental Setup}
\label{sec:methods}

We implemented VaRTOS as a series of architecture-independent extensions to FreeRTOS~\cite{freertos}, a popular open-source real-time operating system. FreeRTOS provides typical operating system abstractions such as preemptive scheduling of multiple tasks, synchronization primitives, and dynamic memory allocation with low overhead and small memory footprint. For our evaluation, we use the TI Stellaris LM3S6965 port of FreeRTOS. The LM3S6965 is a microcontroller based on an ARM Cortex-M3 core, and is representative of the low-power platforms targeted by VaRTOS.

VaRTOS relies on a temperature and instance-dependent power model to perform its optimizations and requires appropriate sensors from its underlying hardware platform to build this model. Temperature sensors are typically embedded into most sensing platforms. Energy consumption and power in various processor modes may be measured directly (e.g. as in~\cite{leap}), or indirectly estimated from remaining battery capacity (e.g. with a ``smart'' battery or as in~\cite{Lachenmann}). Low-cost probes for variability vectors (including aging, frequency, and leakage power) may be embedded into processor cores, and exposed to software as digital counters~\cite{Chan:2012}. 

We evaluate VaRTOS with a series of case study applications under different hardware instances and deployment scenarios (temperature profiles) across a lifetime of 1 year. Because it would be impractical to physically deploy these applications, we rely on VarEMU~\cite{varemu}, a variability-aware virtual machine monitor.

VarEMU is an extension to the QEMU virtual machine monitor~\cite{qemu} that serves as a framework for the evaluation of variability-aware software techniques. VarEMU provides users with the means to emulate variations in power consumption in order to sense and adapt to these variations in software. In VarEMU, timing and cycle count information is extracted from the code being emulated. This information is fed into a variability model, which takes configurable parameters to determine energy consumption in the virtual machine. Through the use (and dynamic change) of parameters in the power model, users can create virtual machines that feature both static and dynamic variations in power consumption. A software stack for VarEMU provides virtual energy monitors to the operating system and processes. With the exception of the driver that interfaces with the VarEMU energy counters, VaRTOS running in VarEMU is unmodified from its version that runs on physical hardware.  

When starting VarEMU, we provide a configuration file with parameters for the power model described in Section~\ref{sec:variability}. For most test cases, we evaluate the system with three instances (nominal, best-case, and worst-case) as shown in Figure~\ref{fig:power}. When necessary for the evaluation, further instances are generated according to SPICE simulation results as described in~\ref{sec:variability}. We also provide a trace of temperature based on hourly temperature data from the National Climactic Data Center~\cite{ncdc} for three locations: Mauna Loa, HI (`best-case': mild temperature, very little variation), Sioux Falls, SD (`nominal-case': average temperature and variation), and Death Valley, CA (`worst-case': extreme temperature and variation). For every hour elapsed on the Virtual Machine, VarEMU reads a new line from the temperature trace file and changes the temperature parameter in the power model accordingly. In order to accelerate the simulation (which would otherwise run in real-time), we use a time scale of 1:3600, resulting in a total simulation time of approximately 2.5 hours for a lifetime of one year.




