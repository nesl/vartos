\section{Conclusion}
\label{sec:conclusion}
We developed an architecture for maximizing application quality while meeting lifetime and energy constraints in the face of power and temperature variation. We achieve this through application elasticity as defined by a \emph{knob}---a tunable variable that increases the quality of a given task at the expense of increased power consumption.  This knob serves to shape the utility curve of each task as well as offer a means by which the operating system can control the amount of active time and thus power given to individual tasks. We implemented online per-instance power modeling and task modeling in VaRTOS, a series of kernel extensions to the FreeRTOS operating system.  Our simulations using VaRTOS show that we can accurately meet a specified lifetime goal with less than 2\% error in most cases and less than 5\% error in the worst case, whereas had we assumed worst-case power consumption errors would range from 5\% to over 70\%.  We further demonstrated the ease with which a developer can adopt the VaRTOS architecture; very minimal user input is required, and the effects of these inputs can be tested using a graphical task modeling tool.  Finally, we presented case studies for multi-node localization applications using wireless sensor networks, estimation problems using Kalman filtering, and multi-block signal processing applications, illustrating how VaRTOS can increase application quality while maintaining lifetime requirements.  Knobs in VaRTOS represent flexible notions of elasticity---we make the assumption in this work that they have a linear relationship with computational time, but in general the only assumption required is that an increase in knob value will increase utility.  In other words, if more advanced modeling techniques are used, we can extend this notion of knobs to adapting many other system parameters, not just task frequency and duration. The VaRTOS architecture allows traditional, non-adaptive tasks to co-exist with adaptable tasks, and it is therefore suitable for a large class of applications where a notion of elasticity in quality exists.  Finally, all software developed in this project is open-source and can be found at https://github.com/nesl/vartos. 