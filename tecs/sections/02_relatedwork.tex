\section{Related Work}
\label{sec:relatedwork}

Hardware-level approaches to address variability have included statistical design approaches \cite{Neiroukh:2005,Datta:2005,Kang:2006}, post-silicon compensation and correction \cite{Gregg:2007,Khandelwal:2007,Tschanz:2002}, and variation avoidance~\cite{Choi:2004,Bhunia:2007,Ghosh:2007}. Furthermore, variation-aware adjustment of hardware parameters (e.g., voltage and frequency), whether in context of adaptive circuits (e.g., \cite{Borkar:2003,Ghosh:2007,Agarwal:2005}), adaptive micro architectures (e.g., \cite{Sylvester:2006,Ernst:2003,Meng:2006,Tiwari:2007}) or software-assisted hardware power management (e.g., \cite{Dighe:2010,Chandra:2009,Teodorescu:2008}) has been explored extensively in literature. 

While low-level treatment of hardware variation is a necessary step forward, application- and process-level adaptations have proven effective methods for combating variation.  The range of actions that software can take in response to variability includes:  altering the computational load by adjusting task activation; using a different set of hardware resources (e.g. using instructions that avoid a faulty module or minimize use of a power hungry module); changing software parameters (e.g., tuning software-controllable  variables such as voltage/frequency); and changing the code that performs a task, either by dynamic recompilation or through algorithmic choice.  Examples of variability-aware software include video codec adaptation~\cite{Pant:2011}, memory allocation~\cite{Bathen:2012}, procedure hopping~\cite{Rahimi:2012}, and error tolerant applications~\cite{Cho:2012}. In embedded sensing, \cite{matsuda2006} and \cite{garg2007} provide lifetime analyses for wireless sensor networks when considering variability power models, offering insights into what such systems stand to gain from explicit treatment of hardware variation.  Garg et al. estimated that a 37\% system lifetime improvement could be achieved through redundancy efforts that totaled a 20\% increased deployment cost. 

This work attempts to mitigate and exploit variations in power consumption through the management of elasticity in application quality by a variability-aware real-time scheduler.
Energy and longevity management in wireless sensor networks and low power embedded systems in general has long been an active area of research. Most previous work in this field, however, ignores the effects of power variations.
%Traditional power management techniques have assumed that, while idle and active powers vary as a function of temperature, they remain uniform across instances.  
Of these variability-agnostic techniques, many have focused on the tradeoff between energy and utility or performance.  For example, \cite{green2010,ghasemzadeh2012} represent attempts at making quality energy-proportional and tunable.  Specifically, \cite{green2010} introduces an architecture that allows developers to specify multiple versions of functions whereby the operating system can sacrifice quality when possible to reduce computational costs.  Similarly, \cite{ghasemzadeh2012} proposes tunable feature selection for wearable embedded systems, where less accurate feature computation can be used at the cost of inference quality. In real-time systems, \cite{liu1994} represents one of many efforts at using approximate computing to save energy where marginal losses in quality can be afforded. In ECOSystem~\cite{Zeng:2002} and Cinder~\cite{Rumble:2009}, energy resources are periodically distributed to tasks which must spend the resources to perform system calls. In these systems, applications adjust their computational load according to energy availability. Our work differs from the previous approaches in that applications need not manage energy directly but instead expose their elasticity in the form of a variable knob that is controlled by the operating system scheduler. Power consumption characteristics for each individual sensor are learned over time, and the system maximizes quality of service across tasks in a variability-aware fashion. 




This work is closely related to that of \cite{Wanner:2012}.  There, the authors describe a method for calculating a system-wide optimal duty cycle ratio given known models for active and idle power as well as probability density functions for deployment temperatures. Here we provide an extension to the work in \cite{Wanner:2012}, showing methods for online learning of power models and providing notions of utility in multi-task applications.  


