\section{Power Variability}

As fabrication technologies improve and feature sizes decrease, hardware variation plays an increasingly important role in determine the power consumption and therefore lifetime of computer systems.  While the large baseline in active power consumption relative to idle power consumption amortizes these variations to some degree (\cite{wanner2011} cites a 10\% variation in active power while \cite{balaji2012} cites between 7\% and 17\% variation), the low baseline in idle power consumption renders it highly susceptible to fabrication-induced variations (\cite{wanner2011} reports a 14x range in measured idle powers across 10 instances of ARM Cortex M3 processors). 

Power consumption in energy-constrained embedded systems is often dominated by time spent in an idle state, waiting to sense, communicate, or perform a processing task. Consequently, the lifetimes of these systems can be wildly variant due to instance-to-instance variation in idle power.  The result of inaccurate lifetime estimates obtained by improper power models can be one of two things: conservative designs can make sure that lifetime is met by adding a guard band to system duty cycle and thus underperforming, or removing these guard bands will result in higher performance while all subsystems (perhaps nodes in a network) are still running, but upon depletion of energy reserves by those processors with higher power consumption the entire system will again suffer.  A processor-specific duty cycle can be assigned so that all lifetimes are met, but without a strategy for distributing resources to individual tasks on a processor there is no guarantee that these variations are handled elegantly and in a way that maximizes application utility.